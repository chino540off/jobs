\documentclass[11pt, a4paper]{moderncv}

% moderncv themes
\moderncvstyle{classic}												% las opciones de estilo son 'casual' (por omision),'classic', 'oldstyle' y 'banking'
\moderncvcolor{blue}												% opciones de color 'blue' (por omision), 'orange', 'green', 'red', 'purple', 'grey' y 'black'

% character encoding
\usepackage[utf8]{inputenc}											% replace by the encoding you are using
\usepackage[frenchb]{babel}

% adjust the page margins
\usepackage[scale=0.93]{geometry}
\AtBeginDocument{\setlength{\maketitlenamewidth}{13cm}}								% only for the classic theme, if you want
														% to change the width of your name placeholder
														% (to leave more space for your address details
\AtBeginDocument{\recomputelengths}										% required when changes are made to page
														% layout lengths

% personal data
\firstname{Olivier}
\familyname{Détour}
\title{\textbf{Ingénieur système \& télécom}\newline \Large{\textit{> 4 ans d'expérience}}}
\address{84 rue Marcadet}{75018 Paris}										% optional, remove the line if not wanted
\mobile{+33 6 13 15 85 55}											% optional, remove the line if not wanted
%\phone{phone (optional)}											% optional, remove the line if not wanted
%\fax{fax (optional)}												% optional, remove the line if not wanted

\email{detour.olivier@gmail.com}										% optional, remove the line if not wanted
\extrainfo{Permis B}												% optional, remove the line if not wanted

\quote{Curriculum Vitae}											% optional, remove the line if not wanted

%\photo[64pt]{picture}												% '64pt' is the height the picture must be resized
														% to and 'picture' is the name of the picture file;

														% optional, remove the line if not wanted

\nopagenumbers{}												% uncomment to suppress automatic page numbering
														% for CVs longer than one page


%----------------------------------------------------------------------------------
%            content
%----------------------------------------------------------------------------------
\begin{document}
\maketitle

\section{Expérience}
\subsection{Emplois}
\cventry{2009--2013}{Ingénieur système \& télécom}{Thales Communication \& Security}{\textit{Gennevilliers}}{}
{
  \begin{itemize}
    \renewcommand{\labelitemi}{$\bullet$  }
    \item Création d'une abstraction VoIP SIP/H323 et création d'applications indépendantes de la
	pile VoIP
    \item Prototype d'un Forwarder IPv4/IPv6 avancé dans le noyau Linux avec apport de valeurs ajoutées
	Thales comme des modes de diffusion non implémentés (XCAST6, GeoCAST), ou utilisant des mécanismes
	de routage particuliers (algorithme multi-critères)
    \item \textbf{Implémentation d'un algorithme de simplification de topologie multi-critères en C99 puis en C++11}
    \item Implémentation d'un protocole d'émission de trames point \`a multi-points et de découverte
	de voisins sur Ethernet
    \item Implémentation d'une segmentation de paquets niveau 2 avec acquittement dans un NetDevice
	Linux et mecanisme de FEC (Forward Error Correction)
    \item Implémentation des RFCs 6051 (IPv6 Addressing of IPv4/IPv6 Translators) et 6145 (IP/ICMP/v4/v6
	Translation Algorithm) dans le noyau Linux
  \end{itemize}
}
\cventry{2009}{Stage de fin d'étude}{Thales Communication}{\textit{Colombes}}{}
{
  Étude de Linux sur l'architecture Cavium (Architecture dédiée aux performances télécom)
}

\subsection{Tutorat}
\cventry{2011}{Vacataire}{ENSTA}{\textit{Paris}}{}
	      {Encadrement du projet informatique \textit{Mini NetFilter} en langage C pour les étudiants
		de 1\iere\ année}
\cventry{2008}{Assistant 2009}{EPITA}{\textit{Le Kremlin-Bic\^etre}}{}
	      {Encadrement des projets UNIX/C/C++ pour les étudiants de 1\iere\ année: cours,
		animation d'ateliers, soutien, création des sujets, évaluation, et soutenance}


\section{Cursus}
\cventry{2004--2009}{EPITA}{Spécialisation Système, Réseaux et Sécurité}{\textit{Le Kremlin-Bic\^etre}}{}{}
\cventry{2002--2003}{Baccalauréat}{Section Scientifique, option mathématique}{Lycée Georges Dumézil, \textit{Vernon (27)}}{}{}
%\cventry{year--year}{Degree}{Institution}{City}{\textit{Grade}}{Description}  % arguments 3 to 6 are optional

\section{Compétences}
\cvlistitem{\textbf{Programmation avancée en C/C++ depuis 7 ans}, projets EPITA et projets professionnels}
\cvlistitem{\textbf{Programmation télécom dans le noyau Linux}, maitrise du cheminement réseau de Linux}
\cvlistitem{\textbf{Connaissances logicielles en télécom et de sécurité}, connaissances renforcées au sein de Thales}
\cvlistitem{\textbf{Connaissances des architectures matérielles embarquées}, stage et projets sur architecture MIPS
	    (Cavium) et PowerPC (PowerQUICC II et III), tests de performances sur ARM (Freescale P1010, iMX6, et Marvell OMAP 3530)
	    et projet personnel sur Marvell Kirkwood (serveur IPSec)}
\cvlistitem{\textbf{Notions d'architecture logicielle, réseau, et de sécurité}, conception d'architectures de
	    systèmes logiciels avec des contraintes télécom et de sécurité}
\cvlistitem{\textbf{Connaissances de nombreux langages de script}, tels que Python, Perl ou Shell}

\section{Langues}
\cvlanguage{Anglais}{TOEIC 775 points}{courant, plusieurs séjours en pays anglophones}
\cvlanguage{Espagnol}{Bac}{notions}

\end{document}

