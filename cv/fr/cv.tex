\documentclass[11pt, a4paper]{moderncv}

% moderncv themes
%\moderncvtheme[blue]{casual}
\moderncvtheme[blue]{classic}											% optional argument are 'blue' (default),
														% 'orange', 'red', 'green', 'grey' and
														% 'roman' (for roman fonts, instead of
														% sans serif fonts)

% character encoding
\usepackage[utf8]{inputenc}											% replace by the encoding you are using
\usepackage[frenchb]{babel}

% adjust the page margins
\usepackage[scale=0.91]{geometry}
\AtBeginDocument{\setlength{\maketitlenamewidth}{10cm}}								% only for the classic theme, if you want
														% to change the width of your name placeholder
														% (to leave more space for your address details
\AtBeginDocument{\recomputelengths}										% required when changes are made to page
														% layout lengths

% personal data
\firstname{Olivier}
\familyname{Détour}
\birth{Né le 18 septembre 1985}{}
\title{\textbf{Ingénieur système \& télécom} \\\\ \Large{> 3 ans d'expérience}}
\address{84 rue Marcadet}{75018 Paris}										% optional, remove the line if not wanted
\mobile{+336 13 15 85 55}											% optional, remove the line if not wanted
%\phone{phone (optional)}											% optional, remove the line if not wanted
%\fax{fax (optional)}												% optional, remove the line if not wanted

\email{detour.olivier@gmail.com}										% optional, remove the line if not wanted
\extrainfo{Permis B}												% optional, remove the line if not wanted

\quote{Curriculum Vitae}											% optional, remove the line if not wanted

%\photo[64pt]{picture}												% '64pt' is the height the picture must be resized
														% to and 'picture' is the name of the picture file;

														% optional, remove the line if not wanted

\nopagenumbers{}												% uncomment to suppress automatic page numbering
														% for CVs longer than one page


%----------------------------------------------------------------------------------
%            content
%----------------------------------------------------------------------------------
\begin{document}
\maketitle

\section{Expérience}
\subsection{Emplois}
\cventry{2009--2012}{Ingénieur système \& télécom}{Thales Communication}{\textit{Colombes}}{}
{
  \begin{itemize}
    \renewcommand{\labelitemi}{$\bullet$  }
    \item Implémentation d'un protocole d'émission point \`a multipoints et de découverte de voisins
    \item Implémentation d'une segmentation de paquets niveau 2 avec acquittement dans un NetDevice
	Linux
    \item Implémentation des RFCs 6051 (IPv6 Addressing of IPv4/IPv6 Translators) et 6145 (IP/ICMP/v4/v6
	Translation Algorithm) dans le noyau Linux
    \item Création d'une abstraction VoIP SIP/H323 et création d'applications indépendantes de la
	pile VoIP
    \item Prototype Nexium Way: Création d'un routeur Thales implémentant les protocoles RSVP-TE et
	MPLS-TE via réservation SIP
  \end{itemize}
}														% arguments 3 to 6 are optional
\cventry{2009}{Stage de fin d'étude}{Thales Communication}{\textit{Colombes}}{}
{
  Étude de Linux sur l'architecture Cavium (Architecture dédiée aux performances télécom)
}														% arguments 3 to 6 are optional

\subsection{Tutorat}
\cventry{2011}{Vacataire}{ENSTA}{\textit{Paris}}{}
	      {Encadrement du projet informatique \textit{Mini NetFilter} en langage C pour les étudiants
		de 1\iere{} année}		 								% arguments 3 to 6 are optional
\cventry{2008}{Assistant 2009}{EPITA}{\textit{Le Kremlin-Bic\^etre}}{}
	      {Encadrement des projets UNIX/C/C++ pour les étudiants de 1\iere{} année: cours,
		animation d'ateliers, soutien, création des sujets, évaluation, et soutenance}			% arguments 3 to 6 are optional


\section{Cursus}
\cventry{2008}{Spécialisation Système, Réseaux et Sécurité}{EPITA}{\textit{Le Kremlin-Bic\^etre}}{}{}		% arguments 3 to 6 are optional
\cventry{2004--2007}{Tronc commun}{EPITA}{\textit{Le Kremlin-Bic\^etre}}{}{}					% arguments 3 to 6 are optional
\cventry{2003--2004}{1\iere{} année préparatoire}{ESGI}{\textit{Paris}}{}{}					% arguments 3 to 6 are optional
\cventry{2002--2003}{Bac S, option mathématiques}{Lycée Georges Dumézil}{\textit{Vernon (27)}}{}{}		% arguments 3 to 6 are optional
%\cventry{year--year}{Degree}{Institution}{City}{\textit{Grade}}{Description}					% arguments 3 to 6 are optional

%\section{Master thesis}
%\cvline{title}{\emph{Title}}
%\cvline{supervisors}{Supervisors}
%\cvline{description}{\small Short thesis abstract}

\section{Compétences}
\cvlistitem{\textbf{Programmation avancée en C/C++ depuis 7 ans}, projets EPITA (micro noyau Kaneton, compilateur
	    Tiger, shell 42sh) et projets professionnels}
\cvlistitem{\textbf{Programmation télécom dans le noyau Linux}, maitrise du cheminement réseau de Linux}
\cvlistitem{\textbf{Connaissances logicielle en télécom et de sécurité}, connaissances renforcées au sein de Thales}
\cvlistitem{\textbf{Connaissances des architectures matérielles embarquées}, stage et projets sur architecture Cavium
	    (MIPS) et PowerQUICC II et III, tests de performances sur Freescale P1010 et Marvell OMAP 3530
	    et projet personnel sur Marvell Kirkwood (serveur IPSec)}
\cvlistitem{\textbf{Notions d'architecture logicielle, réseau, et de sécurité}, conception d'architectures de
	    systèmes logicielles avec des contraintes de télécom et de sécurité}
\cvlistitem{\textbf{Connaissances de nombreux langages de script}, tels que Python, Perl ou Shell}

\section{Langues}
%\cvlanguage{language 1}{Skill level}{Comment}
\cvlanguage{Anglais}{TOEIC 775 points}{lu, parlé, écrit, plusieurs séjours en pays anglophones}
\cvlanguage{Espagnol}{Bac}{notions}

\end{document}


%% end of file `template_en.tex'.
